% !TEX TS-program = xelatex
% !TEX encoding = UTF-8 Unicode
% !Mode:: "TeX:UTF-8"

\documentclass[margin,line]{resume}
\usepackage{zh_CN-fonts}



\begin{document}

\name{姜楠}
\begin{resume}


\section{联系方式}
\begin{tabular}{@{}p{3in}p{3in}}
计算机学院 & {\it 手机:}  (+86) 131-4651-9692 \\
北京航空航天大学  & {\it 邮箱:}  {\tt nanjiang@buaa.edu.cn} \\
北京,中国         & {\it 博客:} {\tt jiangnanhugo.github.io/blog} \\
\end{tabular}


\section{教育背景}
{\bf 北京航空航天大学}, 北京,  中国\\
在读硕士研究生\ 计算机科学与技术专业 . Advisor:  Wenge, Rong.\\
研究方向: 自然语言处理, 机器学习, 深度学习. \\


{\bf 浙江工业大学}, 浙江, 中国\\
学士\ 计算机科学与技术, 平均绩点: 4.0, 排名: 1/60.


\section{发表论文}

\textbf{Nan Jiang}, Wenger Rong, et al. An Efficient Hierarchical Softmax for Large Vocabulary Language Models[C]. International Conference on World Wide Web, 2017. (submitted)

\textbf{Nan Jiang}, Wenge Rong, et al. Event Trigger Identification with Noise Contrastive Estimation[J]. IEEE/ACM Transactions on Computational Biology and Bioinformatics, 2017.

\textbf{Nan Jiang}, Wenge Rong, et al. Modeling Joint Representation with Tri-Modal DBNs for Query and Question Matching[J]. IEICE Transactions on Information and Systems, 2016.

\textbf{Nan Jiang}, Wenge Rong, et al. An empirical analysis of different sparse penalties for autoencoder in unsupervised feature learning[C]. IJCNN, 2015.

Moyuan Huang, Wenge Rong, Tom Arjannikov, \textbf{Nan Jiang}, Zhang Xiong: Bi-Modal Deep Boltzmann Machine Based Musical Emotion Classification. ICANN, 2016.




\section{实习经历}
{\bf 微软}, 北京, 中国 \\
{\em 研发实习生} \hfill {\bf  2016/12 - 现在}\\
调研语言模型的最新发展和改进,实现基本算法并作报告; 负责NMT模型的调参, 使用sgd, adadelta和 momentum-based 提高模型的BLEU结果.\\

{\bf 网易研发中心}, 北京, 中国 \\
{\em 机器翻译实习生} \hfill {\bf  2016/6 - 2016/12}\\
调研语言模型的最新发展和改进,实现基本算法并作报告; 负责NMT模型的调参, 使用sgd, adadelta和 momentum-based 提高模型的BLEU结果.\\


{\bf 教育部工程中心}, 北京航空航天大学计算机学院\\
{\em 研究生} \hfill {\bf 2015/9 }\\
大四下学期进入该实验室实习, 学习深度学习算法 (DNN/LSTM/CNN), 主要研究 NLP 方向. 期间, 发表两篇会议论文和一篇期刊论文, 参加一次国际会议并获得该会议颁发的 Travel Grant.


\section{ Github 项目}
{神经机器翻译模型:} github.com/jiangnanHugo/nmt;

{语言模型评测:} github.com/jiangnanHugo/language\_modeling;

{基于NCE的事件触发词识别:} github.com/jiangnanHugo/mlee-nce;

{Matlab实现的自学习算法:} github.com/jiangnanHugo/Self-Taught-Learning;

{Pthread \& MPI 矩阵并行乘法:} github.com/jiangnanHugo/parallel-Computing.


\section{IT 技能}

编程语言: Python, C/C++, Java, Matlab;

工具: Bash Script, Git, Vim, theano.


\section{获奖情况}
二等奖学金, 北京航空航天大学, 2016$\sim$2015

Travel Grant of IEEE IJCNN(IEEE), 2015

优秀毕业生, 浙江工业大学, 2015

校一等奖学金, 浙江工业大学, 2014$\sim$2012

国家奖学金, 浙江工业大学, 2012


\end{resume}

\end{document}
