% !TEX TS-program = xelatex
% !TEX encoding = UTF-8 Unicode
% !Mode:: "TeX:UTF-8"

\documentclass{resume}
\RequirePackage{xeCJK}



\setCJKmainfont[
BoldFont=SimHei,
ItalicFont=KaiTi,
SmallCapsFont=SimHei
]{SimSun}
\setCJKsansfont{SimHei}
\setCJKmonofont{Fangsong}

\setCJKfamilyfont{zhsong}{SimSun}
\setCJKfamilyfont{zhhei}{SimHei}
\setCJKfamilyfont{zhfs}{Fangsong}
\setCJKfamilyfont{zhkai}{KaiTi}
\setCJKfamilyfont{zhli}{LiSu}
\setCJKfamilyfont{zhyou}{YouYuan}

\newcommand*{\songti}{\CJKfamily{zhsong}} % 宋体
\newcommand*{\heiti}{\CJKfamily{zhhei}}   % 黑体
\newcommand*{\kaishu}{\CJKfamily{zhkai}}  % 楷书
\newcommand*{\fangsong}{\CJKfamily{zhfs}} % 仿宋
\newcommand*{\lishu}{\CJKfamily{zhli}}    % 隶书
\newcommand*{\youyuan}{\CJKfamily{zhyou}} % 幼圆



\begin{document}
\pagenumbering{gobble} % suppress displaying page number

\name{姜楠}



\section{教育背景}
\begin{tabular}{@{}p{3in}p{3in}}
计算机学院 & {\it Phone:}  (+86) 131-4651-9692 \\
北京航空航天大学  & {\it 邮箱:}  {\tt nanjiang@buaa.edu.cn} \\
北京,中国         & {\it 博客:} {\tt jiangnanhugo.github.io/blog} \\
\end{tabular}


\section{教育背景}
{\bf 北京航空航天大学}, Beijing,  China\\
在读硕士研究生\ 计算机科学与技术专业 . Advisor:  Wenge, Rong.\\
研究方向: 自然语言处理, 机器学习, 深度学习. \\


{\bf 浙江工业大学}, Zhejiang, China\\
学士\ 计算机科学与技术, 平均绩点: 4.0, 排名: 1/60.


\section{发表论文}
\begin{itemize}
\item \textbf{Nan Jiang}, Wenger Rong, et al. An Efficient Hierarchical Softmax for Large Vocabulary Language Models[C]. International Conference on World Wide Web, 2017. (submitted)
\item \textbf{Nan Jiang}, Wenge Rong, et al. Event Trigger Identification with Noise Contrastive Estimation[J]. IEEE/ACM Transactions on Computational Biology and Bioinformatics, 2017.

\item \textbf{Nan Jiang}, Wenge Rong, et al. Modeling Joint Representation with Tri-Modal DBNs for Query and Question Matching[J]. IEICE Transactions on Information and Systems, 2016.

\item \textbf{Nan Jiang}, Wenge Rong, et al. An empirical analysis of different sparse penalties for autoencoder in unsupervised feature learning[C]. IJCNN, 2015.

\item Moyuan Huang, Wenge Rong, Tom Arjannikov, \textbf{Nan Jiang}, Zhang Xiong: Bi-Modal Deep Boltzmann Machine Based Musical Emotion Classification. ICANN, 2016.
\end{itemize}

\section{ Github 项目}
\begin{itemize}
  \item {神经机器翻译模型:} github.com/jiangnanHugo/nmt;
  \item {语言模型评测:} github.com/jiangnanHugo/language\_modeling;
  \item {基于NCE的事件触发词识别:} github.com/jiangnanHugo/mlee-nce;
  \item {Matlab实现的自学习算法:} github.com/jiangnanHugo/Self-Taught-Learning;
  \item {Pthread \& MPI 矩阵并行乘法:} github.com/jiangnanHugo/parallel-Computing.
\end{itemize}

\section{实习经历 }
{\bf 北京网易研发中心}, Beijing, China

{\em 机器翻译实习生} \hfill {\bf June, 2016 - Dec, 2016}\\
调研语言模型的最新发展和改进,实现基本算法并作报告; 负责NMT模型的调参, 使用sgd, adadelta和 momentum-based 提高模型的BLEU结果.


{\bf 教育部工程中心}, 北京航空航天大学计算机学院

{\em 研究生} \hfill {\bf Sep, 2015 }\\
大四下学期进入该实验室实习, 学习深度学习算法 (DNN/LSTM/CNN), 主要研究 NLP 方向. 期间, 发表两篇会议论文和一篇期刊论文, 参加一次国际会议并获得该会议颁发的 Travel Grant.

\section{IT 技能}
% increase linespacing [parsep=0.5ex]
\begin{itemize}
  \item 编程语言: Python, C/C++, Java, Matlab;
  \item 工具: Bash Script, Git, Vim, theano.
\end{itemize}

\section{获奖情况}
二等奖学金, 北京航空航天大学, 2016$\sim$2015

Travel Grant of IEEE IJCNN(IEEE), 2015

优秀毕业生, 浙江工业大学, 2015

校一等奖学金, 浙江工业大学, 2014$\sim$2012

国家奖学金, 浙江工业大学, 2012




\end{document}
