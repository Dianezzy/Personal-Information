% !TEX TS-program = xelatex
% !TEX encoding = UTF-8 Unicode
% !Mode:: "TeX:UTF-8"

\documentclass[margin,line]{resume}
\usepackage{zh_CN-fonts}
\usepackage[hidelinks]{hyperref}



\begin{document}

\name{姜楠}
\begin{resume}


\section{联系方式}
\begin{tabular}{@{}p{3in}p{3in}}
计算机学院 & { 手机:}  (+86) 131-4651-9692 \\
北京航空航天大学  & { 邮箱:}  {\tt nanjiang@buaa.edu.cn} \\
北京,中国         & { 博客:} {\tt jiangnanhugo.github.io/blog} \\
\end{tabular}


\section{教育背景}
{\bf 北京航空航天大学}, 北京,  中国\\
在读硕士研究生\ 计算机科学与技术专业 . \\
研究方向: 自然语言处理, 机器学习, 深度学习. \\


{\bf 浙江工业大学}, 浙江, 中国\\
学士\ 计算机科学与技术, 平均绩点: 4.0, 排名: 1/60.


\section{发表论文}

\textbf{Nan Jiang}, Wenge Rong, Min Gao, Yikang Shen and Zhang Xiong. Exploration of Tree-based Hierarchical Softmax for Recurrent Language Models[C]. Proceedings of the Twenty-Sixth International Joint Conference on Artificial Intelligence (IJCAI), 2017. [\href{https://github.com/jiangnanHugo/lmkit}{code}][\href{https://www.ijcai.org/proceedings/2017/0271.pdf}{pdf}]

\textbf{Nan Jiang}, Wenge Rong, Yifan Nie, Yikang Shen and Zhang Xiong. Event Trigger Identification with Noise Contrastive Estimation[J]. IEEE/ACM Transactions on Computational Biology and Bioinformatics, 2017. [\href{https://github.com/jiangnanHugo/mlee-nce}{code}], [\href{https://github.com/jiangnanhugo/paper/blob/master/APBC2017/APBC2017.pdf}{pdf}]

\textbf{Nan Jiang}, Wenge Rong, Baolin Peng, Yifan Nie and Zhang Xiong. Modeling Joint Representation with Tri-Modal DBNs for Query and Question Matching[J]. IEICE Transactions on Information and Systems, 2016.

\textbf{Nan Jiang}, Wenge Rong, Baolin Peng, Yifan Nie and Zhang Xiong. An empirical analysis of different sparse penalties for autoencoder in unsupervised feature learning[C]. International Joint Conference on Neural Networks (IJCNN), 2015.

Yikang Shen, Wenge Rong, \textbf{Nan Jiang}, Baolin Peng Jie Tang and Zhang Xiong. Word Embedding Based Correlation Model for Question/Answer Matching[C]. Proceedings of the Thirtieth {AAAI} Conference on Artificial Intelligence (AAAI), 2017.

Yifan Qian, Wenge Rong, \textbf{Nan Jiang}, Jie Tang and Zhang Xiong. Citation regression analysis of computer science publications in different ranking categories and subfields[J]. Scientometrics 110(3): 1351-1374 (2017).




\section{实习经历}
{\bf 微软}, 北京, 中国 \\
{\em 研发实习生} \hfill {\bf  2016/12 - 现在}\\
整理并清洗聊天对话数据集,调试seq2seq模型,并对当前的论文中的改进模型加以复现和评价。\\

{\bf 网易研发中心}, 北京, 中国 \\
{\em 机器翻译实习生} \hfill {\bf  2016/6 - 2016/12}\\
调研语言模型的最新发展和改进,实现基本算法并作报告; 负责NMT模型的参数优化。\\




\section{比赛}
KDD Cup: 车流量预测比赛 2017, 第三名.

Kaggle: Quora 句子相似检测, 2017. 第八名


\section{IT 技能}
\begin{tabular}{@{}p{3in}p{3in}}
语言: CUDA, C/C++, Python, Java, Matlab;& 工具: Git, Vim, Markdown, Bash; \\
DL框架: Theano, Lasagne, Tensorflow, Keras  &  \\
\end{tabular}



\section{获奖情况}
二等奖学金, 北京航空航天大学, 2016$\sim$2015

Travel Grant of IEEE IJCNN(IEEE), 2015

优秀毕业生, 浙江工业大学, 2015

校一等奖学金, 浙江工业大学, 2014$\sim$2012

国家奖学金, 浙江工业大学, 2012


\end{resume}

\end{document}
