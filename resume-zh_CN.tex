% !TEX TS-program = xelatex
% !TEX encoding = UTF-8 Unicode
% !Mode:: "TeX:UTF-8"

\documentclass{resume}


\usepackage{zh_CN-fonts} 

\usepackage{cite}


\begin{document}
\pagenumbering{gobble} % suppress displaying page number

\name{姜楠}

% {E-mail}{mobilephone}{homepage}
% be careful of _ in emaill address
\contactInfo{nanjiang@buaa.edu.cn}{(+86) 131-4651-9692}{\href{https://www.cnblogs.com/ZJUT-jiangnan/}{个人博客 (博客园)}}
% {E-mail}{mobilephone}
% keep the last empty braces!
%\contactInfo{xxx@yuanbin.me}{(+86) 131-221-87xxx}{}

\section{\faGraduationCap\  教育背景}
\datedsubsection{\textbf{北京航空航天大学}}{2015 - 至今}
在读硕士研究生\ 计算机科学与技术专业 .
\datedsubsection{\textbf{浙江工业大学}}{2011 - 2015}
学士\ 计算机科学与技术, 平均绩点: 4.0, 排名: 1/60.

\section{\faLightbulbO\ 研究方向}
自然语言处理, 机器学习, 深度学习.


\section{\faBook\ 发表论文 }
\begin{itemize}[parsep=0.5ex]
\item \textbf{Nan Jiang}, Wenger Rong, et al. An Efficient Hierarchical Softmax for Large Vocabulary Language Models[C]. International Conference on World Wide Web, 2017. (submitted)
\item \textbf{Nan Jiang}, Wenge Rong, et al. Event Trigger Identification with Noise Contrastive Estimation[J]. IEEE/ACM Transactions on Computational Biology and Bioinformatics, 2017.

\item \textbf{Nan Jiang}, Wenge Rong, et al. Modeling Joint Representation with Tri-Modal DBNs for Query and Question Matching[J]. IEICE Transactions on Information and Systems, 2016.

\item \textbf{Nan Jiang}, Wenge Rong, et al. An empirical analysis of different sparse penalties for autoencoder in unsupervised feature learning[C]. IJCNN, 2015.

\item Moyuan Huang, Wenge Rong, Tom Arjannikov, \textbf{Nan Jiang}, Zhang Xiong: Bi-Modal Deep Boltzmann Machine Based Musical Emotion Classification. ICANN, 2016.
\end{itemize}

\section{\faGithub\ Github 项目}
\begin{itemize}[parsep=0.5ex]
  \item \href{https://github.com/jiangnanHugo/nmt}{神经机器翻译模型:} github.com/jiangnanHugo/nmt;
  \item \href{https://github.com/jiangnanHugo/language_modeling}{语言模型评测:} github.com/jiangnanHugo/language\_modeling;
  \item \href{https://github.com/jiangnanHugo/mlee-nce}{基于NCE的事件触发词识别:} github.com/jiangnanHugo/mlee-nce;
  \item \href{https://github.com/jiangnanHugo/Self-Taught-Learning}{Matlab实现的自学习算法:} github.com/jiangnanHugo/Self-Taught-Learning;
  \item \href{https://github.com/jiangnanHugo/parallel-Computing}{Pthread \& MPI 矩阵并行乘法:} github.com/jiangnanHugo/parallel-Computing.
\end{itemize}

\section{\faUsers\ 实习经历 }
\datedsubsection{\textbf{北京网易研发中心}}{2016.06}
\role{有道事业部}{机器翻译实习生}
调研语言模型的最新发展和改进,实现基本算法并作报告; 负责NMT模型的调参, 使用sgd, adadelta和 momentum-based 提高模型的BLEU结果.

\datedsubsection{\textbf{北京航空航天大学计算机学院}}{2015.09}
\role{教育部工程中心}{研究生}
大四下学期进入该实验室实习, 学习深度学习算法 (DNN/LSTM/CNN), 主要研究 NLP 方向. 期间, 发表两篇会议论文和一篇期刊论文, 参加一次国际会议并获得该会议颁发的 Travel Grant.



\section{\faCogs\ IT 技能}
% increase linespacing [parsep=0.5ex]
\begin{itemize}[parsep=0.5ex]
  \item 编程语言: Python, C/C++, Java, Matlab;
  \item 工具: Bash Script, Git, Vim, theano.
\end{itemize}

\section{\faStar\ 获奖情况}
\datedline{二等奖学金(北京航空航天大学)}{2016$\sim$2015}
\datedline{Travel Grant of IEEE IJCNN(IEEE)}{2015}
\datedline{优秀毕业生(浙江工业大学)}{2015}
\datedline{校一等奖学金(浙江工业大学)}{2014$\sim$2012}
\datedline{国家奖学金(浙江工业大学)}{2012}




\end{document}
